\documentclass[12pt]{article}

%% preamble: Keep it clean; only include those you need
\usepackage{amsmath}
\usepackage[margin = 1in]{geometry}
\usepackage{graphicx}
\usepackage{booktabs}
\usepackage{natbib}

% for space filling
\usepackage{lipsum}
% highlighting hyper links
\usepackage[colorlinks=true, citecolor=blue]{hyperref}


%% meta data

\title{Proposal: Association Between Air Pollution and Mortality}
\author{Patrick Brogan\\
  Department of Statistics\\
  University of Connecticut
}

\begin{document}
\maketitle


\paragraph{Introduction}
For my paper, I will be analyzing the association between air pollution and different types of mortality. Specifically, I will be researching the association between fine particulate matter (PM2.5) and mortality from cardiovascular disease, non-communicable respiratory disease, and malignant neoplasms. Previous research has demonstrated such an association in specific localities or municipalities\citep{dockery1993association}\citep{sunyer1996air}\citep{jerrett2005spatial}, and this paper seeks to build on this research.

\paragraph{Specific Aims}
The aims of my research paper will be to analyze the impact of air pollution on global public health by focusing on the association between PM2.5 pollution and mortality from cardiovascular disease, non-communicable respiratory disease, and malignant neoplasms. Specifically, I wish to analyze the impact of PM2.5 pollution on mortality caused by rheumatic heart disease, hypertensive heart disease, cerebrovascular disease, inflammatory heart diseases, chronic obstructive pulmonary disease, asthma, and trachea, bronchus, and lung cancers.

\paragraph{Data}
I will use the World Health Organization's Air Quality Database for my data on PM2.5 pollution prevalence by country. I will also use the World Health Organization's Mortality Database for my data on the prevalence of mortality from different types of cardiovascular illnesses, non-communicable respiratory illnesses, and malignant neoplasms. I will use the sex and age filters in the mortality database to control for sex and age covariates.

\paragraph{Research Design and Methods}
For my research design, using the data from the WHO, I will conduct a linear regression analysis to test for correlations between PM2.5 pollution and mortality from the aforementioned specific types of cardiovascular and respiratory illnesses. I will test for collinearity and covariance between mortality, age, and sex by using the model
\[
  Y = \beta_{0} + \beta_{1}X + \beta_{2}AGE + \beta_{3}SEX + \beta_{4}(X)(AGE) + \beta_{5}(X)(SEX) + \beta_{6}(X)(AGE)(SEX) + E
\]

\paragraph{Discussion}
\lipsum[5]

\paragraph{Conclusion}
\lipsum[6]

\bibliographystyle{chicago}

\end{document}
