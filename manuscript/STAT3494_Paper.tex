\documentclass[12pt, letterpaper, twoside]{article}\usepackage[]{graphicx}\usepackage[]{xcolor}
% maxwidth is the original width if it is less than linewidth
% otherwise use linewidth (to make sure the graphics do not exceed the margin)
\makeatletter
\def\maxwidth{ %
  \ifdim\Gin@nat@width>\linewidth
    \linewidth
  \else
    \Gin@nat@width
  \fi
}
\makeatother

\definecolor{fgcolor}{rgb}{0.345, 0.345, 0.345}
\newcommand{\hlnum}[1]{\textcolor[rgb]{0.686,0.059,0.569}{#1}}%
\newcommand{\hlstr}[1]{\textcolor[rgb]{0.192,0.494,0.8}{#1}}%
\newcommand{\hlcom}[1]{\textcolor[rgb]{0.678,0.584,0.686}{\textit{#1}}}%
\newcommand{\hlopt}[1]{\textcolor[rgb]{0,0,0}{#1}}%
\newcommand{\hlstd}[1]{\textcolor[rgb]{0.345,0.345,0.345}{#1}}%
\newcommand{\hlkwa}[1]{\textcolor[rgb]{0.161,0.373,0.58}{\textbf{#1}}}%
\newcommand{\hlkwb}[1]{\textcolor[rgb]{0.69,0.353,0.396}{#1}}%
\newcommand{\hlkwc}[1]{\textcolor[rgb]{0.333,0.667,0.333}{#1}}%
\newcommand{\hlkwd}[1]{\textcolor[rgb]{0.737,0.353,0.396}{\textbf{#1}}}%
\let\hlipl\hlkwb

\usepackage{framed}
\makeatletter
\newenvironment{kframe}{%
 \def\at@end@of@kframe{}%
 \ifinner\ifhmode%
  \def\at@end@of@kframe{\end{minipage}}%
  \begin{minipage}{\columnwidth}%
 \fi\fi%
 \def\FrameCommand##1{\hskip\@totalleftmargin \hskip-\fboxsep
 \colorbox{shadecolor}{##1}\hskip-\fboxsep
     % There is no \\@totalrightmargin, so:
     \hskip-\linewidth \hskip-\@totalleftmargin \hskip\columnwidth}%
 \MakeFramed {\advance\hsize-\width
   \@totalleftmargin\z@ \linewidth\hsize
   \@setminipage}}%
 {\par\unskip\endMakeFramed%
 \at@end@of@kframe}
\makeatother

\definecolor{shadecolor}{rgb}{.97, .97, .97}
\definecolor{messagecolor}{rgb}{0, 0, 0}
\definecolor{warningcolor}{rgb}{1, 0, 1}
\definecolor{errorcolor}{rgb}{1, 0, 0}
\newenvironment{knitrout}{}{} % an empty environment to be redefined in TeX

\usepackage{alltt}

\usepackage{amsmath}
\usepackage{booktabs}
\usepackage{amsthm}
\usepackage[tight,footnotesize]{subfigure}
\usepackage{graphicx}
\usepackage{filecontents}
\usepackage{parskip}
\usepackage{listings}
\usepackage[margin=1.2in]{geometry}
\usepackage[colorlinks=true]{hyperref}
\hypersetup{colorlinks = true, linkcolor = blue, citecolor=blue, urlcolor = blue}
\usepackage{natbib}
\usepackage{enumitem}
\usepackage{setspace}
\graphicspath{ {/Users/patrickbrogan/Desktop/Manuscript/} }

\usepackage[pagewise]{lineno}
%\linenumbers*[1]
% %% patches to make lineno work better with amsmath
\newcommand*\patchAmsMathEnvironmentForLineno[1]{%
 \expandafter\let\csname old#1\expandafter\endcsname\csname #1\endcsname
 \expandafter\let\csname oldend#1\expandafter\endcsname\csname end#1\endcsname
 \renewenvironment{#1}%
 {\linenomath\csname old#1\endcsname}%
 {\csname oldend#1\endcsname\endlinenomath}}%
\newcommand*\patchBothAmsMathEnvironmentsForLineno[1]{%
 \patchAmsMathEnvironmentForLineno{#1}%
 \patchAmsMathEnvironmentForLineno{#1*}}%

 \AtBeginDocument{%
 \patchBothAmsMathEnvironmentsForLineno{equation}%
 \patchBothAmsMathEnvironmentsForLineno{align}%
 \patchBothAmsMathEnvironmentsForLineno{flalign}%
 \patchBothAmsMathEnvironmentsForLineno{alignat}%
 \patchBothAmsMathEnvironmentsForLineno{gather}%
 \patchBothAmsMathEnvironmentsForLineno{multline}%
}

\doublespacing

\setlength\parindent{24pt}

\title{Country-Level Association Between Particulate Matter Air Pollution and Mortality}
\author{Patrick Brogan\\[1ex]
  Department of Statistics, University of Connecticut\\}
\IfFileExists{upquote.sty}{\usepackage{upquote}}{}
\begin{document}

\maketitle

\begin{abstract}
Air pollution has been consistently shown to be associated with mortality,
specifically respiratory morbidity. Identifying associations between specific
air pollutants and specific types of morbidity can inform future public health
practice and public environmental policy globally. The aim of this paper was to
build on this research with a country-level analysis of the global effect of
PM2.5 pollution on three types of mortality: asthma, rheumatic heart disease,
and hypertensive heart disease. Using a linear mixed effects model to control
for covariates such as time, GDP per capita, and sex, this paper found a
significant positive association between ambient PM2.5 concentration and the age
standardized death rate from asthma, but a significant negative association
between ambient PM2.5 concentration and the age standardized death rate from
heart disease in each country.
\end{abstract}

\section*{Introduction}
\addcontentsline{toc}{section}{Introduction}
Air pollution is a serious public health and environmental issue, having been
linked to over 6.5 million global deaths each year \citep{fuller2022pollution}.
Previous research has demonstrated an association between air pollution and
mortality in specific localities or municipalities\citep{dockery1993association,
sunyer1996air, jerrett2005spatial, analitis2006short}. When talking specifically
about particulate matter, ambient particulate matter pollution is attributable
to 4.14 million (95\% CI: 3.45–4.8 million) global deaths in 2019 alone
\citep{fuller2022pollution}. This opens up the question of what types of
mortality are most heavily associated with ambient PM2.5 pollution. Some studies
have demonstrated an association between ambient particulate matter pollution
and respiratory and cardiovascular deaths \citep{analitis2006short}. However,
other studies have found no significant relationship between particulate matter
pollution and the morbidity and mortality rate \citep{khojasteh2021long}.\par

The aim of this study is to analyze
the association between ambient PM2.5 air pollution in ppm and mortality from
asthma, rheumatic heart disease, and hypertensive heart disease at a country-wide
level rather than at the municipal level. Though previous literature analyzing
global air pollution and mortality has found significant associations between
the two, there has not yet been an analysis focusing specifically on ambient
PM2.5 air pollution and mortality from asthma, rheumatic heart disease, and
hypertensive heart disease globally. According to the Lancet Commission on
pollution and health \citep{khojasteh2021long}, there is a substantial gap
between males and females in the annual mortality due to ambient PM2.5 pollution,
as 1.70 million (1.38–2.01) females died	in 2019 versus 2.44 million
(2.02–2.83 million) males; this indicated that sex would need to be controlled
for. Using a linear mixed effects model to account for the effects of sex, time,
and GDP per capita, this study aims to analyze the association between ambient
PM2.5 concentration and mortality from asthma, hypertensive heart disease, and
rheumatic heart disease.

\section*{Data}
\addcontentsline{toc}{section}{Data}
The data for this analysis was compiled from the WHO's Ambient Air Quality
Database \citep{who2022airquality}, the WHO's Mortality Database
\citep{who2022all}, and the World Bank's data on GDP per capita, PPP
in constant 2017 international dollars \citep{worldbank2022}.\par
The primary sources
for the air quality data were official reports of countries, official national
and subnational reports on measurements of PM10 or PM2.5 and ground measurements
compiled in the framework of the Global Burden of Disease project. Data reported
by Clean Air for Asia, the Air quality e-reporting database of the European
Environment Agency for Europe, the AirNow Programme from the United States
embassies and consulates, and values from peer-reviewed journals were also used.\par
The data on mortality comprise deaths registered in national vital registration
systems, with underlying cause of death as coded by the relevant national
authority. These data are official national statistics in the sense that they
have been transmitted to the World Health Organization by the competent
authorities of the countries concerned.\par
The data on each country's GDP per
capita was compiled by the World Bank using each country's national accounts and
census data, using either the 1993 or 2008 System of National Accounts
Methodology. Since the data on ambient particulate matter pollution ranged from
2010 to 2016, this was the period that was analyzed for all datasets. The WHO
datasets on air quality and mortality help to answer for associations between
ambient PM2.5 air pollution and mortality from asthma, rheumatic heart disease,
and hypertensive heart disease.\par
These datasets were uploaded into RStudio and
were combined and ordered by corresponding year and country name so that analyses
could be more easily conducted. This combined dataframe was used to conduct this
regression analysis. Since three types of mortality were analyzed, there were
three subsets of data with the responses variables for each being mortality
rate from asthma, rheumatic heart disease, and hypertensive heart disease.


\section*{Methods}
\addcontentsline{toc}{section}{Methods}

\begin{figure}[t]
\includegraphics[scale=0.6]{boxcox.pdf}
\centering
\caption{Box-Cox Normal Probability Plot}
\label{fig:Figure 1}
    \vspace{1cm}
\end{figure}
\begin{figure}[t]
\includegraphics[scale=0.5]{scatterplotmatrix.asthma.pdf}
\centering
\caption{Scatter plot matrix for asthma dataset (Value is PM2.5 concentration)}
\label{fig:Figure 2}
    \vspace{1cm}
\end{figure}

The three datasets were compiled into one dataframe in RStudio where all
statistical analyses were conducted. Since the air quality database only included
data from 2010 to 2016, only that time period was analyzed. The parameter estimated
is age standardized mortality rate in a given country from one of the three
specified mortalities and the explanatory variable is average country-wide
ambient PM2.5 concentration in ppm with the covariates being sex, time, and GDP
per capita. Instead of using a simple linear regression model, this study tests
for associations between nationwide PM2.5 air concentration and age standardized
mortality rates by using a linear mixed effects model which accounts for the
effects of sex, time, and GDP per capita such that:
\[
  MOR = \beta_{0} + \beta_{1}POL + \beta_{2}SEX + \beta_{3}TIME +\beta_{4}GDP
  + \beta_{5}(POL)(SEX) + \beta_{6}(POL)(TIME)
\]
\[
 + \beta_{7}(POL)(GDP) + E
\]

I did this using the lmer function in RStudio using the lme4 package in RStudio.
Running this model for the compiled dataframe yielded the point estimates for
the slopes and intercepts as well as the standard error for each. I was able to
find the p-values for each explanatory variable using the formula
\[
  p-value = 2(1 - \phi(|t-value|))
\]

\begin{figure}[t]
\includegraphics[scale=0.5]{scatterplotmatrix.rheum.pdf}
\centering
\caption{Scatter plot matrix for rheumatic heart disease dataset (Value is PM2.5 concentration)}
\label{fig:Figure 3}
    \vspace{1cm}
\end{figure}

\begin{figure}[t]
\includegraphics[scale=0.5]{scatterplotmatrix.hyper.pdf}
\centering
\caption{Scatter plot matrix for hypertensive heart disease dataset (Value is PM2.5 concentration)}
\label{fig:Figure 4}
    \vspace{1cm}
\end{figure}

\begin{figure}[t]
\includegraphics[scale=0.5]{influenceplot.pdf}
\centering
\caption{Influence plot of mixed effects models for asthma (top), rheumatic heart
disease (middle), and hypertensive heart disease (bottom)}
\label{fig:Figure 5}
    \vspace{1cm}
\end{figure}

where \begin{math}\phi(Z)\end{math} is the CDF of a standard normal random
variable. In order to perform this regression analysis, regression diagnostics
had to be conducted.\par

First, the normality assumption was checked using the
Shapiro-Wilk Test of Normality. Performing this test on the dataset containing
death rates from asthma yielded a test statistic of W = 0.83355
(\begin{math}p<\end{math}2.2e-16). Performing this test on the dataset containing
death rates from rheumatic heart disease yielded a test statistic of W = 0.64097
(\begin{math}p<\end{math}2.2e-16). Performing this test on the dataset containing
death rates from hypertensive heart disease yielded a test statistic of
W = 0.32637 (\begin{math}p<\end{math}2.2e-16). This indicated that a power
transformation of the response variable may be necessary. Applying the Box-Cox
transformation to each of the mixed effects models yieled a power estimate of
\begin{math}\lambda=0.02754374\end{math} (\autoref{fig:Figure 1}). Using this
power transformation, a new linear mixed effects modxel was derived:

\[
  MOR^{\lambda} = \beta_{0} + \beta_{1}POL + \beta_{2}SEX + \beta_{3}TIME +\beta_{4}GDP
  + \beta_{5}(POL)(SEX) + \beta_{6}(POL)(TIME)
\]
\[
 + \beta_{7}(POL)(GDP) + E
\]

\par Then, the equal variance assumption was analyzed among the datasets. The
Bartlett test was used to test for heteroskedasticity between the predictor
variables (except for sex) and the response variable. For the asthma dataset,
the Bartlett test indicated heteroscadesticity between the response, age
standardized death rate per 100,000, and all the predictors. For the rheumatic
heart disease dataset,the Bartlett test indicated heteroscadesticity between the
response and all the predictors except for year. For the hypertensive heart
disease dataset, the Bartlett test indicated heteroscadesticity between the
response and all the predictors. Since the data violated the homoscadesticity
assumption, different weights were needed for the predictor variables. The weights
were derived by first calculating the residuals and fitted values for each regression
model, where fitted values are defined as

\[
  \hat{MOR_i} = \hat{\beta_{0}} + \hat{\beta_{1}}POL_i + \hat{\beta_{2}}SEX_i +
  \hat{\beta_{3}}TIME_i +\hat{\beta_{4}}GDP_i + \hat{\beta_{5}}(POL_i)(SEX_i)
\]
\[
+ \hat{\beta_{6}}(POL_i)(TIME_i) + \hat{\beta_{7}}(POL_i)(GDP_i)
\]

and residuals are \begin{math}\epsilon_i = MOR_i^{\lambda}-\hat{MOR_i^{\lambda}}
\end{math}. Once these values were calculated, a simple linear regression between
the two was of the form \begin{math}|\epsilon| = \gamma_0 + \gamma_1\hat{MOR^
{\lambda}}\end{math} was derived, for which fitted values were found using the
formula

\[
\hat{|\epsilon_i|} = \hat{\gamma_0} + \hat{\gamma_1}\hat{MOR_i^
{\lambda}}
\]

\par Weights for each observation were calculated using the formula

\[
weight = \frac{1}{\hat{|\epsilon_i|}^2}
\]

and then these weights were factored into the linear mixed effects model for
each morbidity.
\par We then tested the linearity assumption by plotting partial
residual plots for all of the predictor and response variables
(\autoref{fig:Figure 2}, \autoref{fig:Figure 3}, \autoref{fig:Figure 4}). Based
on the partial residual plots, the linearity assumption was not noticeably
violated by the data. Lastly, we looked for influential observations which may
skew the data. Lastly, we tested for influential observations using Cook's
distance (\autoref{fig:Figure 5}). Using Bonferroni's outlier test, studentized
residuals with Bonferroni \begin{math}p < \end{math}0.05 were found in each
dataset and those observations were removed from the analysis. The final models
used were equivalent to the general code:

\begin{lstlisting}[language=R]
final <- lmer(Death.rate^lambda ~ PM2.5
+ GDP.per.capita + Period + Sex
+ (0 + GDP.per.capita|PM2.5)
+ (0 + Sex|PM2.5)
+ (0 + Period|PM2.5),
data, weights)
\end{lstlisting}

\section*{Results}
\addcontentsline{toc}{section}{Results}
Using the linear mixed effect model with the given covariates, point estimates
for the slopes of each explanatory variable were obtained as well as the standard
error for each. Utilizing the given formula for the p-value, the significance of
each explanatory variable in predicting the response variable, age standardized
mortality rate per 100,000, was analyzed. After all processes were
conducted in order to control for covariates, significant associations were
found between a given country's average ambient PM2.5 air concentration and
the mortality rate from asthma, rheumatic heart disease, or hypertensive heart
disease.\par


\begin{table}[h!]
\centering
\begin{tabular}{|c | c c c c|}
\toprule
Parameter & Estimate & Standard Error & T-value & P-value \\ [0.5ex]
\midrule
(Intercept) &	2.1357 &	0.6466 &	3.3030 &	0.0010 \\
PM2.5 Concentration &	0.0003 &	0.0001 &	3.7653 &	0.0002 \\
GDP per capita &	-5.63e-07	 &	0.0000 &	-9.9249 &	0.0000 \\
Period &	-0.0006 &	0.0003 &	-1.7396 &	0.0819 \\
Sex &	0.0014 &	0.0011 &	1.2621 &	0.2069 \\ [1ex]
\bottomrule
\end{tabular}
\label{tab:table1}
\caption{Significance of explanatory variables on age standardized death rate
  from asthma in mixed effects model. Sex and Time (Period) aren't significant
  predictors but PM2.5 pollution (Value) and GDP per capita are at the
  \begin{math}\alpha = 0.05\end{math} significance level.}
\end{table}

\begin{table}[h!]
\centering
\begin{tabular}{|c | c c c c|}
\toprule
Parameter & df & SSR & MSR & F \\ [0.5ex]
\midrule
PM2.5 Concentration & 1 & 0.0150351 & 0.0150351 & 47.0453 \\
GDP per capita & 1 & 0.0314527 & 0.0314527 & 98.4167 \\
Period & 1 & 0.0009657 & 0.0009657 & 3.0216 \\
Sex & 1 & 0.0005090 & 0.0005090 & 1.5928 \\ [1ex]
\bottomrule
\end{tabular}
\label{tab:table2}
\caption{ANOVA table for regression of PM2.5 concentration on mortality from
asthma}
\end{table}

\begin{table}[h!]
\centering
\begin{tabular}{|c | c c c c|}
\toprule
Parameter & Estimate & Standard Error & T-value & P-value \\ [0.5ex]
\midrule
Intercept &	2.0642 &	0.7159 &	2.8834 &	0.0039 \\
PM2.5 Concentration &	-0.0002 &	0.0001 &	-2.7671 &	0.0057 \\
GDP per capita &	-1.97e-07		 &	0.0000 &	-4.7579 &	0.0000 \\
Period &	-0.0005 &	0.0004 &	-1.4798 &	0.1389 \\
Sex &	-0.0053 &	0.0007 &	-7.3443 &	0.0000 \\[1ex]
\bottomrule
\end{tabular}
\label{tab:table3}
  \caption{Significance of explanatory variables on age standardized death rate
  from rheumatic heart disease in mixed effects model. Time (Period) isn't a
  significant predictor but PM2.5 pollution (Value), sex and GDP per capita are
  at the \begin{math}\alpha = 0.05\end{math} significance level.}
\end{table}

\begin{table}[h!]
\centering
\begin{tabular}{|c | c c c c|}
\toprule
Parameter & df & SSR & MSR & F \\ [0.5ex]
\midrule
Value & 1 & 0.002132 & 0.002132 & 2.4405 \\
GDP.per.capita & 1 & 0.019769 & 0.019769 & 22.6307 \\
Period & 1 & 0.001969 & 0.001969 & 2.2540 \\
Sex & 1 & 0.047117 & 0.047117 & 53.9387 \\ [1ex]
\bottomrule
\end{tabular}
\label{tab:table4}
\caption{ANOVA table for regression of PM2.5 concentration on mortality from
rheumatic heart disease}
\end{table}

\begin{table}[h!]
\centering
\begin{tabular}{|c | c c c c|}
\toprule
Parameter & Estimate & Standard Error & T-value & P-value \\ [0.5ex]
\midrule
Intercept &	1.1360 &	0.9242 &	1.2291 &	0.2190 \\
PM2.5 Concentration &	-0.0003 &	0.0001 &	-2.5896 &	0.0096 \\
GDP per capita &	-4.61e-07 &	0.0000 &	-8.9911 &	0.0000 \\
Period &	-0.0001 &	0.0005 &	-0.1336 &	0.8937 \\
Sex &	0.0718  &	0.0015 &	47.6318 &	0.0000 \\ [1ex]
\bottomrule
\end{tabular}
\label{tab:table5}
  \caption{Significance of explanatory variables on age standardized death rate
  from hypertensive heart disease in mixed effects model. Time (Period) isn't a
  significant predictor but PM2.5 pollution (Value), sex and GDP per capita are
  at the \begin{math}\alpha = 0.05\end{math} significance level.}
\end{table}

\begin{table}[h!]
\centering
\begin{tabular}{|c | c c c c|}
\toprule
Parameter & df & SSR & MSR & F \\ [0.5ex]
\midrule
PM2.5 Concentration & 1 & 0.0000 & 0.0000 & 0.0051 \\
GDP per capita & 1 & 0.2924 & 0.2924 & 104.1244 \\
Period & 1 & 0.0000 & 0.0000 & 0.0015 \\
Sex & 1 & 6.3712 & 6.3712 & 2268.7877 \\ [1ex]
\bottomrule
\end{tabular}
\label{tab:table6}
\caption{ANOVA table for regression of PM2.5 concentration on mortality from
hypertensive heart disease}
\end{table}

The mixed effect model regression for asthma yielded a significant
positive association between PM2.5 pollution and the death rate from asthma
(F=47.0453, p=0.0002, \autoref{tab:table2}), which is less
than the significance level of \begin{math}\alpha = 0.05\end{math}. In that
regression, GDP per capita was a significant predictor of country-level death rate
from asthma (\begin{math}p<\end{math}0.0001), though time and sex weren't
significant predictors, as p-values of 0.0819 and 0.2069 were yielded
respectively. In contrast, the mixed effect model regression for rheumatic heart
disease yielded a significant negative association between PM2.5 pollution and
the death rate from asthma (F=2.4405, p=0.0057, \autoref{tab:table4}), which is
less than the significance level of \begin{math}\alpha = 0.05\end{math}. In that
regression, GDP per capita and sex were significant predictors, as p-values of
\begin{math}p<\end{math}0.0001 were yielded for both, though time was not a
significant predictor (p=0.1389). Likewise, the mixed effect model regression for
hypertensive heart disease yielded a significant negative association between
PM2.5 pollution and the death rate from asthma (F=0.0051, p=0.0096,
\autoref{tab:table6}), which is greater than the significance level of
\begin{math}\alpha = 0.05\end{math}. In that regression, GDP per capita and sex
were significant predictors, as p-values of \begin{math}p<\end{math}0.0001 were
yielded for both, though time was not a significant predictor (p=0.8937).

\section*{Discussion}
The association between air pollution and mortality has been the topic of
scienticic inquiry for a long time, and many studies and analyses have been
conducted in order to elucidate the association between the two
\citep{dockery1993association, sunyer1996air, jerrett2005spatial, analitis2006short}.
The fact that air pollution contributes to over 6.5 million deaths every year
\citep{fuller2022pollution} makes this topic all the more consequential for
informing global public health and public policy practices. As mentioned before,
the research pertaining to ambient PM2.5 air pollution's association with
mortality has been equivocal, with some studies finding a significant association
\citep{analitis2006short} and others finding no such significant association
\citep{khojasteh2021long}. This study falls in the former column, as
significant associations were demonstrated between a country's average ambient
PM2.5 air concentration and their age standardized death rate per 100,000 from
asthma, rheumatic heart disease, and hypertensive heart disease.\par
This study has several limitations. Firstly, data was only collected on the
average ambient PM2.5 concentration at the country level, and did not
differentiate between PM2.5 concentration levels in different regions or
localities. Similarly, the data collected on mortality rates in each country
only provided an average age standardized death rate for the entirety of each
country in the analysis. Due to the data having a much more broad and less
specific scope, the outcomes of the analyses may be affected. Secondly, the data
did not provide information about population density, so population density
was not accounted for in the analysis. Lastly, the data didn't include years
more recent than 2016, and this makes it so the results are not as relevant to
the current year, though the presence of COVID-19 may have skewed the mortality
rates from asthma, rheumatic heart disease or hypertensive heart disease. Despite
these limitations, this study adds to the plethora of literature that exists on
the subject of air pollution and mortality, helping to inform public health and
public policy in the future.

\end{document}
